% Arquivo LaTeX de exemplo de dissertação/tese a ser apresentados à CPG do IME-USP
%
% Versão 6: Sex Nov  10 18:00:00 BRT 2017
%
% Criação: Jesús P. Mena-Chalco
% Revisão: Fabio Kon e Paulo Feofiloff
% Adaptação para UTF8, biblatex e outras melhorias: Nelson Lago


%%%%%%%%%%%%%%%%%%%%%%%%%%%%%%%%%%%%%%%%%%%%%%%%%%%%%%%%%%%%%%%%%%%%%%%%%%%%%%%%
%%%%%%%%%%%%%%%%%%%%%%%%%%%%%%% PREÂMBULO LaTeX %%%%%%%%%%%%%%%%%%%%%%%%%%%%%%%%
%%%%%%%%%%%%%%%%%%%%%%%%%%%%%%%%%%%%%%%%%%%%%%%%%%%%%%%%%%%%%%%%%%%%%%%%%%%%%%%%

% Este pacote gera avisos durante a compilação sobre comandos
% considerados obsoletos.
\RequirePackage[l2tabu, orthodox]{nag}

% "Book" tem capítulos (e partes, mas normalmente não usamos) e, se o documento
% é frente-e-verso, cada capítulo começa em uma página de numeração ímpar.
% Report é similar, mas cada capítulo começa em uma nova página, par ou ímpar.
% É possível mudar esse comportamento com a opção "openany". Observe que você
% pode adaptar este modelo para escrever artigos, mudando a classe do
% documento de "book" para "article" ou a classe de algum periódico específico.
%
% A opção frente-e-verso aqui significa, por exemplo, que as margens das páginas
% ímpares e pares são diferentes ou que números de página aparecem à direita
% ou à esquerda alternadamente. Nada impede que você crie um documento "só
% frente" e, ao imprimir, faça a impressão frente-e-verso.
%
% Aqui também definimos a língua padrão do documento e línguas adicionais. A
% classe em si não usa essa informação mas, passando as opções de língua aqui,
% elas são repassadas para todas as demais classes, e diversas classes mudam
% seu comportamento em função da língua (em especial, babel/polyglossia).
% A última língua da lista é a língua padrão do documento.
\documentclass[11pt,twoside,brazil,english]{book}
%\documentclass[11pt,twoside,english,brazil]{book}
%\documentclass[11pt,twoside,english,brazil]{article}

% Vários pacotes e opções de configuração genéricos; para personalizar o
% resultado, modifique este arquivo.
\input meta/miolo-preambulo

% O arquivo com os dados bibliográficos; você pode executar este comando
% mais de uma vez para acrescentar múltiplos arquivos
\addbibresource{references.bib}


%%%%%%%%%%%%%%%%%%%%%%%%%%%%%%%%%%%%%%%%%%%%%%%%%%%%%%%%%%%%%%%%%%%%%%%%%%%%%%%%
%%%%%%%%%%%%%%%%%%%%%%% METADADOS (TÍTULO, AUTOR ETC.) %%%%%%%%%%%%%%%%%%%%%%%%%
%%%%%%%%%%%%%%%%%%%%%%%%%%%%%%%%%%%%%%%%%%%%%%%%%%%%%%%%%%%%%%%%%%%%%%%%%%%%%%%%

% Estes comandos definem o título e autoria do trabalho e devem sempre ser
% definidos, pois além de serem utilizados para criar a capa (tanto no estilo
% do IME quanto com o comando padrão \maketitle), também são armazenados nos
% metadados do PDF
\title{Ranking Source Code Static Analysis Warnings for Continuous Monitoring of Free/Libre and Open Source Software Repositories}
\author{Athos Coimbra Ribeiro}

% O pacote hyperref armazena alguns metadados no PDF gerado (em particular,
% o conteúdo de "\title" e "\author"). Também é possível armazenar outros
% dados, como uma lista de palavras-chave.
\hypersetup{
  pdfkeywords={master thesis, source code static analysis, IME/USP},
}

% Para gerar a capa e demais páginas iniciais de uma tese/dissertação para o
% IME/USP automaticamente no formato sugerido, modifique os dados neste arquivo.
% Se estiver escrevendo um artigo ou for gerar a capa etc. manualmente, remova.
\input meta/metadados-tese


%%%%%%%%%%%%%%%%%%%%%%%%%%%%%%%%%%%%%%%%%%%%%%%%%%%%%%%%%%%%%%%%%%%%%%%%%%%%%%%%
%%%%%%%%%%%%%%%%%%%%%%% AQUI COMEÇA O CONTEÚDO DE FATO %%%%%%%%%%%%%%%%%%%%%%%%%
%%%%%%%%%%%%%%%%%%%%%%%%%%%%%%%%%%%%%%%%%%%%%%%%%%%%%%%%%%%%%%%%%%%%%%%%%%%%%%%%

\begin{document}

% Se estiver usando a classe "article" ao invés de "book", não existem os
% comandos "frontmatter", "mainmatter" etc. abaixo. Além disso, para gerar o
% título, você pode usar o comando padrão do LaTeX "\maketitle"
%\maketitle

%%%%%%%%%%%%%%%%%%%%%%%%%%% CAPA E FOLHAS DE ROSTO %%%%%%%%%%%%%%%%%%%%%%%%%%%%%

% Aqui vai o conteúdo inicial que aparece antes do capítulo 1, ou seja,
% página de rosto, abstract, TOC etc. O comando frontmatter faz números
% de página aparecem em algarismos romanos ao invés de arábicos e
% desabilita a contagem de capítulos (ele não existe na classe "article")
\frontmatter

% Este formato está definido mais acima na seção "APARÊNCIA/FORMATAÇÃO"
\pagestyle{frontback}

% As folhas de rosto no formato sugerido para teses/dissertações do IME/USP.
% Se estiver escrevendo um artigo ou não quiser usar, remova.
\input meta/folhas-de-rosto

% Sumário (obrigatório)
\tableofcontents

% Referências indiretas ("x", veja "y") para o índice remissivo (opcionais,
% pois o índice é opcional). É comum colocar esses itens no final do documento,
% junto com o comando \printindex, mas em alguns casos isso torna necessário
% executar texindy (ou makeindex) mais de uma vez, então colocar aqui é melhor.
\index{Inglês|see{Língua estrangeira}}
\index{Figuras|see{Floats}}
\index{Tabelas|see{Floats}}
\index{Código-fonte|see{Floats}}
\index{Subcaptions|see{Subfiguras}}
\index{Sublegendas|see{Subfiguras}}
\index{Equações|see{Modo Matemático}}
\index{Fórmulas|see{Modo Matemático}}
\index{Rodapé, notas|see{Notas de rodapé}}
\index{Captions|see{Legendas}}
\index{Versão original|see{Tese/Dissertação!versões}}
\index{Versão corrigida|see{Tese/Dissertação!versões}}
\index{Palavras estrangeiras|see{Língua estrangeira}}

%%%%%%%%%%%%%%%%%%%%%%%%%%%%%%%% CAPÍTULOS %%%%%%%%%%%%%%%%%%%%%%%%%%%%%%%%%%%%%

% Aqui vai o conteúdo principal do trabalho, ou seja, os capítulos que compõem
% a dissertação/tese. O comando mainmatter reinicia a contagem de páginas,
% modifica a numeração para números arábicos e ativa a contagem de capítulos
% (ele não existe na classe "article")
\mainmatter

% Este formato está definido mais acima na seção "APARÊNCIA/FORMATAÇÃO"
% e só funciona com book/report, pois usa o nome dos capítulos nos cabeçalhos;
% se estiver usando article, comente ou troque por "plain"
\pagestyle{mainmatter}

% No texto principal, vamos usar espaçamento entre linhas padrão
\singlespacing
%\onehalfspacing

% Os capítulos de compõem a dissertação/tese, com numeração normal, podem
% ser inseridos diretamente aqui ou "puxados" de outros arquivos
\input chapters/introduction

% Um parágrafo em LaTeX termina com uma linha vazia; como não é possível
% ter certeza que um arquivo incluído (neste caso, "paginas-iniciais")
% terminou com uma linha vazia, é recomendável usar o comando "par" para
% garantir que o último parágrafo realmente terminou.
\par

\input chapters/literature-review
\par

\input chapters/kiskadee
\par

\input chapters/ranking
\par

\input chapters/results
\par

\input chapters/conclusion
\par


%%%%%%%%%%%%%%%%%%%%%%%%%%%%%%%% APÊNDICES %%%%%%%%%%%%%%%%%%%%%%%%%%%%%%%%%%%%%

% Aqui vão apêndices. O comando appendix reinicia a numeração de capítulos e
% passa a numerá-los com letras. Como os anteriores, ele não existe na classe
% "article".
\appendix

% Este formato está definido mais acima na seção "APARÊNCIA/FORMATAÇÃO"
\pagestyle{appendix}

% Os apêndices podem ser inseridos diretamente aqui ou "puxados" de outros
% arquivos
%\input ape-conjuntos
\input appendices/literature.tex
\par
\input appendices/implementation.tex
\par
\input appendices/firehose.rng.tex
\par

%%%%%%%%%%%%%%%%%%%%%%%%%%%%%% SEÇÕES FINAIS %%%%%%%%%%%%%%%%%%%%%%%%%%%%%%%%%%%

% Aqui vão a bibliografia, índice remissivo e outras seções similares.
% O comando backmatter desabilita a numeração de capítulos e também não existe
% na classe "article".
\backmatter

% Este formato está definido mais acima na seção "APARÊNCIA/FORMATAÇÃO"
\pagestyle{frontback}

\singlespacing   % espaçamento simples

% A bibliografia é obrigatória

%%%%%%%%% Bibliografia com natbib (preterido): %%%%%%%%%
%\bibliographystyle{alpha-ime}% citação bibliográfica alpha
%\bibliographystyle{plainnat-ime} % citação bibliográfica textual
%\bibliography{bibliografia}  % associado ao arquivo: 'bibliografia.bib'

%%%%%%%% Bibliografia com biblatex (preferido): %%%%%%%%

\printbibliography[
  %title=Bibliografia\label{bibliografia},
  % Inclui a bibliografia no sumário
  heading=bibintoc,
]

% imprime o índice remissivo no documento (opcional)
\printindex

\end{document}
