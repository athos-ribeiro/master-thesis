%%%%%%%%%%%%%%%%%%%%%%%%%%%%%%%%%%%%%%%%%%%%%%%%%%%%%%%%%%%%%%%%%%%%%%%%%%%%%%%%
%%%%%%%%%%%%%%%%%%%%%%%%%%%%% METADADOS DA TESE %%%%%%%%%%%%%%%%%%%%%%%%%%%%%%%%
%%%%%%%%%%%%%%%%%%%%%%%%%%%%%%%%%%%%%%%%%%%%%%%%%%%%%%%%%%%%%%%%%%%%%%%%%%%%%%%%

% Este pacote define o formato da capa, páginas de rosto, dedicatória e
% resumo. Se você pretende criar essas páginas manualmente, não precisa
% carregar este pacote nem definir os dados abaixo.
\usepackage{imeusp}

% Define o texto da capa e da referência que vai na página do resumo
\mestrado
%\doutorado

% Se "\title" está em inglês, você pode definir o título em português aqui
\tituloport{Título do trabalho}

% Se "\title" está em português, você pode definir o título em inglês aqui
\tituloeng{Ranking Source Code Static Analysis Warnings for Continuous Monitoring of Free/Libre and Open Source Software Repositories}

% Se o trabalho não tiver subtítulo, basta remover isto.
%\subtitulo{um subtítulo}

% Se isto não for definido, "\subtitulo" é utilizado no lugar
%\subtituloeng{a subtitle}

\orientador{Prof. Dr. Fabio Kon}

% Se não houver, remova
%\coorientador{Prof. Dr. Ciclano}

\programa{Ciência da Computação}

% Se isto não for definido, "\programa" é utilizado no lugar
\programaeng{Computer Science}

% Se não houver, remova
\apoio{Durante o desenvolvimento deste trabalho o autor recebeu auxílio
financeiro da Capes}

% Se isto não for definido, "\apoio" é utilizado no lugar
\apoioeng{During this work, the author was supported by CAPES}

\localdefesa{São Paulo}

\datadefesa{10 de Abril de 2018}

% Se isto não for definido, "\datadefesa" é utilizado no lugar
\datadefesaeng{April 10th, 2018}

% Necessário para criar a referência do documento que aparece
% na página do resumo
\ano{2018}

\banca{
  \begin{itemize}
    \item Profª. Drª. Nome Completo (orientadora) - IME-USP [sem ponto final]
    \item Prof. Dr. Nome Completo - IME-USP [sem ponto final]
    \item Prof. Dr. Nome Completo - IMPA [sem ponto final]
  \end{itemize}
}

% Se isto não for definido, "\banca" é utilizado no lugar
\bancaeng{
  \begin{itemize}
    \item Prof. Dr. Nome Completo (advisor) - IME-USP [sem ponto final]
    \item Prof. Dr. Nome Completo - IME-USP [sem ponto final]
    \item Prof. Dr. Nome Completo - IMPA [sem ponto final]
  \end{itemize}
}

% Palavras-chave separadas por ponto e finalizadas também com ponto.
\palavraschave{Análise estática de código-fonte. Qualidade de software. Falsos positivos.}

\keywords{Source code static analysis. Software quality. False positives.}

% Se quiser estabelecer regras diferentes, converse com seu
% orientador
\direitos{Autorizo a reprodução e divulgação total ou parcial
deste trabalho, por qualquer meio convencional ou
eletrônico, para fins de estudo e pesquisa, desde que
citada a fonte.
\\
I authorize reproduction and disclosure of this work through any
conventional or eletronic means for study and research purposes, provided that
the source is mentioned.}

% Isto deve ser preparado em conjunto com o bibliotecário
%\fichacatalografica{
% nome do autor, título, etc.
%}
