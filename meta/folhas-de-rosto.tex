%%%%%%%%%%%%%%%%%%%%%%%%%%% CAPA E FOLHAS DE ROSTO %%%%%%%%%%%%%%%%%%%%%%%%%%%%%

% Este formato está definido mais acima na seção "APARÊNCIA/FORMATAÇÃO"
\pagestyle{plain}

% Nas seções iniciais, vamos usar espaçamento entre linhas 1,5
\onehalfspacing

% Embora as páginas iniciais *pareçam* não ter numeração, a numeração existe,
% só não é impressa. O comando \mainmatter (mais abaixo) reinicia a contagem
% de páginas e elas passam a ser impressas. Isso significa que existem duas
% páginas com o número "1": a capa e a página do primeiro capítulo. O pacote
% hyperref não lida bem com essa situação. Assim, vamos desabilitar hyperlinks
% para números de páginas no início do documento e reabilitar mais adiante.
\hypersetup{pageanchor=false}

% A capa; o parâmetro pode ser "port" ou "eng" para definir a língua
%\capaime[port]
\capaime[eng]

% Se você não quiser usar a capa padrão, você pode criar uma outra
% capa manualmente ou em um programa diferente. No segundo caso, é só
% importar a capa como uma página adicional usando o pacote pdfpages.
%\includepdf{./arquivo_da_capa.pdf}

% A página de rosto da versão para depósito (ou seja, a versão final
% antes da defesa) deve ser diferente da página de rosto da versão
% definitiva (ou seja, a versão final após a incorporação das sugestões
% da banca). Os parâmetros podem ser "port/eng" para a língua e
% "provisoria/definitiva" para o tipo de página de rosto.
%\pagrostoime[port]{definitiva}
%\pagrostoime[port]{provisoria}
%\pagrostoime[eng]{definitiva}
\pagrostoime[eng]{provisoria}

%%%%%%%%%%%%%%%%%%%% DEDICATÓRIA, RESUMO, AGRADECIMENTOS %%%%%%%%%%%%%%%%%%%%%%%

% A definição deste ambiente está no pacote imeusp; se você não
% carregar esse pacote, precisa cuidar desta página manualmente.
\begin{dedicatoria}
  Para Gabi.
\end{dedicatoria}

% Após a capa e as páginas de rosto, começamos a numerar as páginas; com isso,
% podemos também reabilitar links para números de páginas no pacote hyperref.
% Isso porque, embora contagem de páginas aqui começe em 1 e no primeiro
% capítulo também, o fato de uma numeração usar algarismos romanos e a outra
% algarismos arábicos é suficiente para evitar problemas.
\pagenumbering{roman}
\hypersetup{pageanchor=true}

% Agradecimentos:
% Se o candidato não quer fazer agradecimentos, deve simplesmente eliminar
% esta página. A epígrafe, obviamente, é opcional; é possível colocar
% epígrafes em todos os capítulos. O comando "\chapter*" faz esta seção
% não ser incluída no sumário.
\chapter*{Agradecimentos}
%\epigrafe{Do. Or do not. There is no try.}{Mestre Yoda}

I would like to thank my advisor, professor Fabio Kon, whose support was
essential for all my research activities. It had been an honor to learn how
research should be performed in computer science from Fabio.  Professor Paulo
Roberto Miranda Meirelles for his support and for the trust and confidence he
has placed on me through the years. Dr. Paul E. Black and the Software
Assurance Metrics and Tool Evaluation (SAMATE) team for all their patience and
teachings during my stay at the US National Institute of Standards and
Technology, which inspired my work. The Fedora Project community, for receiving
our ideas in the Google Summer of Code project and giving us feedback on our
work. We would also like to thank Nelson Lago for all his ideas and wonderful
reviews.

Athos Ribeiro was funded by CAPES during this research extent.

David Silva became a core contributor to kiskadee code base through the Google
Summer of Code program under the Fedora Project organization mentored by Athos
Ribeiro.

The following undergraduate students of University of Brasília contributed to
kiskadee's source code and developed most of the current system front end, as of
the date this document was written. They did so so under a maintenance course
taught by Prof. Paulo Roberto Miranda Meirelles.
They are Adailson Santos,
Daniel Moura,
Danilo Barros,
David Carlos,
Eduardo Gomes,
Eduardo Nunes,
Eduardo Quintino,
Fabio Teixeira,
Gabriel Climaco,
Gesiel Freitas,
Lucas Andrade,
Matheus Godinho,
Matheus Mello,
Maxwell Oliveira,
Omar Junior,
Rafael Rabetti,
Thiago Moreira,
Vitor Bertulucci, and
Vitor Borges.
Thank you for your valuable contributions.

Finally, I would like to thank my parents, Edivaldo and Rita, for always
understanding and supporting me unconditionally.

% O resumo é obrigatório, em português e inglês. Este comando também gera
% automaticamente a referência para o próprio documento, conforme as normas
% sugeridas da USP
\begin{resumo}{port}

Embora exista grande variedade de analisadores estáticos de código-fonte
disponíveis no mercado, tanto com licenças proprietárias, quanto com licenças
livres, cada uma dessas ferramentas mostra melhor desempenho em um pequeno
conjunto de problemas distinto, dificultando a escolha de uma única ferramenta de
análise estática para analisar um programa. A combinação das análises de
diferentes ferramentas pode reduzir o número de falsos negativos, mas gera um
aumento no número de falsos positivos (que já é alto para muitas dessas
ferramentas). Uma solução interessante é filtrar esses resultados para
identificar os problemas com menores probabilidades de serem falsos
positivos.
Este trabalho apresenta \textit{kiskadee}, um sistema para promover o uso da
análise estática de código fonte durante o ciclo de desenvolvimento de
software provendo relatórios de análise estática ranqueados. Primeiramente,
\textit{kiskadee} roda diversos analisadores estáticos no código-fonte. Em seguida,
utilizando um modelo de classificação, os potenciais bugs detectados pelos
analisadores estáticos são ranqueados conforme sua importância, onde defeitos
críticos são colocados no topo de uma lista, e potenciais falsos positivos,
ao fim da mesma lista.
Para treinar o modelo de classificação do \textit{kiskadee}, realizamos uma
pós-análise nos relatórios gerados por três analisadores estáticos ao
analisarem casos de teste sintéticos disponibilizados pelo \textit{National
Institute of Standards and Technology} (NIST) dos Estados Unidos. Para tornar
a técnica apresentada o mais genérica possível, limitamos nossos dados às
informações contidas nos relatórios de análise estática das três ferramentas,
não utilizando outras informações, como históricos de mudança ou métricas
extraídas do código-fonte dos programas inspecionados. As características
extraídas desses relatórios foram utilizadas para treinar um conjunto de
árvores de decisão utilizando o algoritmo AdaBoost para gerar um
classificador mais forte, atingindo uma acurácia de classificação de 0,8 (a
taxa de falsos positivos das ferramentas utilizadas foi de 0,61, quando
combinadas). Finalmente, utilizamos esse classificador para ranquear os
alarmes dos analisadores estáticos nos baseando na probabilidade de um dado
alarme ser de fato um bug no código-fonte. Resultados experimentais
mostram que, em média, quando inspecionando alarmes ranqueados pelo \textit{kiskadee},
encontram-se $5,2$ vezes menos falsos positivos antes de se encontrar cada bug
quando a mesma inspeção é realizada para uma lista ordenada de forma aleatória.

\end{resumo}

% O resumo é obrigatório, em português e inglês. Este comando também gera
% automaticamente a referência para o próprio documento, conforme as normas
% sugeridas da USP
\begin{resumo}{eng}
While there is a wide variety of both open source and proprietary source code
static analyzers available in the market, each of them usually performs
better in a small set of problems, making it hard to choose one single
tool to rely on when examining a program. Combining the analysis of
different tools may reduce the number of false negatives, but yields a
corresponding increase in the number of false positives (which is
already high for many tools). An interesting solution, then, is to
filter these results to identify the issues least likely to be false
positives.
This work presents \textit{kiskadee}, a system to support the usage of static
analysis during software development by providing carefully ranked static
analysis reports.  First, it runs multiple static analyzers on the source
code. Then, using a classification model, the potential bugs detected by the
static analyzers are ranked based on their importance, with critical flaws
ranked first, and potential false positives ranked last.
To train kiskadee's classification model, we post-analyze the reports generated
by three tools on synthetic test cases provided by the US National Institute
of Standards and Technology. To make our technique as general as
possible, we limit our data to the reports themselves, excluding other
information such as change histories or code metrics. The features extracted
from these reports are used to train a set of decision trees using AdaBoost
to create a stronger classifier, achieving 0.8 classification accuracy (the
combined false positive rate from the used tools was 0.61).  Finally, we use
this classifier to rank static analyzer alarms based on the probability of a
given alarm being an actual bug. Our experimental results show that, on
average, when inspecting warnings ranked by \textit{kiskadee}, one hits $5.2$
times less false positives before each bug than when using a randomly sorted
warning list.
\end{resumo}

%%%%%%%%%%%%%%%%%%%%%%%%%%% LISTAS DE FIGURAS ETC. %%%%%%%%%%%%%%%%%%%%%%%%%%%%%

% Todas as listas são opcionais; Usando "\chapter*" elas não são incluídas
% no sumário. As listas geradas automaticamente também não são incluídas
% por conta das opções "notlot" e "notlof" que usamos mais acima.

% Listas criadas manualmente
\chapter*{List of Abbreviations}
\begin{tabular}{rl}
  API & Application Programming Interface \\
  CWE & Common Weakness Enumeration \\
  CVE & Common Vulnerabilities and Exposures \\
  FLOSS & Free/Libre and Open Source Software \\
  FN & False Negative \\
  FP & False Positive \\
  NIST & National Institute of Standards and Technology \\
  ROC & Receiver Operating Characteristic \\
  SAMATE & Software Assurance Metrics And Tool Evaluation \\
  SATE & Static Analysis Tool Exposition \\
  SIG & Fedora Special Interest Group \\
  TN & True Negative \\
  TP & True Positive \\
  UI & User Interface \\
  UnB & University of Brasília

\end{tabular}

% Normalmente, "\chapter*" faz o novo capítulo iniciar em uma nova página.
% Como cada uma destas listas é muito curta, não faz muito sentido fazer
% isso aqui. "\let\clearpage\relax" é um "truque sujo" para temporariamente
% desabilitar a quebra de página.

%\chapter*{Lista de Símbolos}
%{\let\cleardoublepage\relax \addvspace{55pt plus 15pt minus 15pt} \chapter*{Lista de Símbolos} }
%\begin{tabular}{rl}
        %$\omega$    & Frequência angular\\
        %$\psi$      & Função de análise \emph{wavelet}\\
        %$\Psi$      & Transformada de Fourier de $\psi$\\
%\end{tabular}

% Listas criadas automaticamente
\listoffigures
%{\let\cleardoublepage\relax \addvspace{55pt plus 15pt minus 15pt} \listoffigures }

\listoftables
%{\let\cleardoublepage\relax \addvspace{55pt plus 15pt minus 15pt} \listoftables }
