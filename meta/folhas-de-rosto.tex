%%%%%%%%%%%%%%%%%%%%%%%%%%% CAPA E FOLHAS DE ROSTO %%%%%%%%%%%%%%%%%%%%%%%%%%%%%

% Este formato está definido mais acima na seção "APARÊNCIA/FORMATAÇÃO"
\pagestyle{plain}

% Nas seções iniciais, vamos usar espaçamento entre linhas 1,5
\onehalfspacing

% Embora as páginas iniciais *pareçam* não ter numeração, a numeração existe,
% só não é impressa. O comando \mainmatter (mais abaixo) reinicia a contagem
% de páginas e elas passam a ser impressas. Isso significa que existem duas
% páginas com o número "1": a capa e a página do primeiro capítulo. O pacote
% hyperref não lida bem com essa situação. Assim, vamos desabilitar hyperlinks
% para números de páginas no início do documento e reabilitar mais adiante.
\hypersetup{pageanchor=false}

% A capa; o parâmetro pode ser "port" ou "eng" para definir a língua
%\capaime[port]
\capaime[eng]

% Se você não quiser usar a capa padrão, você pode criar uma outra
% capa manualmente ou em um programa diferente. No segundo caso, é só
% importar a capa como uma página adicional usando o pacote pdfpages.
%\includepdf{./arquivo_da_capa.pdf}

% A página de rosto da versão para depósito (ou seja, a versão final
% antes da defesa) deve ser diferente da página de rosto da versão
% definitiva (ou seja, a versão final após a incorporação das sugestões
% da banca). Os parâmetros podem ser "port/eng" para a língua e
% "provisoria/definitiva" para o tipo de página de rosto.
%\pagrostoime[port]{definitiva}
%\pagrostoime[port]{provisoria}
%\pagrostoime[eng]{definitiva}
\pagrostoime[eng]{provisoria}

%%%%%%%%%%%%%%%%%%%% DEDICATÓRIA, RESUMO, AGRADECIMENTOS %%%%%%%%%%%%%%%%%%%%%%%

% A definição deste ambiente está no pacote imeusp; se você não
% carregar esse pacote, precisa cuidar desta página manualmente.
\begin{dedicatoria}
Esta seção é opcional e fica numa página separada; ela pode ser usada para
uma dedicatória ou epígrafe.
\end{dedicatoria}

% Após a capa e as páginas de rosto, começamos a numerar as páginas; com isso,
% podemos também reabilitar links para números de páginas no pacote hyperref.
% Isso porque, embora contagem de páginas aqui começe em 1 e no primeiro
% capítulo também, o fato de uma numeração usar algarismos romanos e a outra
% algarismos arábicos é suficiente para evitar problemas.
\pagenumbering{roman}
\hypersetup{pageanchor=true}

% Agradecimentos:
% Se o candidato não quer fazer agradecimentos, deve simplesmente eliminar
% esta página. A epígrafe, obviamente, é opcional; é possível colocar
% epígrafes em todos os capítulos. O comando "\chapter*" faz esta seção
% não ser incluída no sumário.
\chapter*{Agradecimentos}
\epigrafe{Do. Or do not. There is no try.}{Mestre Yoda}

Athos Ribeiro Was funded by CAPES during this research extent.

David Silva bacame a core contributor to kiskadee code base through the Google
Summer of Code program under the Fedora Project organization mentored by Athos
Ribeiro.

The following undergraduate students of University of Brasília contributed to
kiskadee's source code and developed most of the current system front end, as of
the date this document was written. They did so so under a maitainance course
taught by Dr. Paulo Roberto Miranda Meirelles.

They are Adailson Santos,
Daniel Moura,
Danilo Barros,
David Carlos,
Eduardo Gomes,
Eduardo Nunes,
Eduardo Quintino,
Fabio Teixeira,
Gabriel Climaco,
Gesiel Freitas,
Lucas Andrade,
Matheus Godinho,
Matheus Mello,
Maxwell Oliveira,
Omar Junior,
Rafael Rabetti,
Thiago Moreira,
Vitor Bertulucci, and
Vitor Borges.
Thank you for your valuable contributions.

Finally, we would also like to thank Nelson Lago for all his ideas and reviews.

% O resumo é obrigatório, em português e inglês. Este comando também gera
% automaticamente a referência para o próprio documento, conforme as normas
% sugeridas da USP
\begin{resumo}{port}
  \textbf{Tradução a ser realizada após revisão do abstract em inglês}
%Elemento obrigatório, constituído de uma sequência de frases concisas e
%objetivas, em forma de texto.  Deve apresentar os objetivos, métodos empregados,
%resultados e conclusões.  O resumo deve ser redigido em parágrafo único, conter
%no máximo 500 palavras e ser seguido dos termos representativos do conteúdo do
%trabalho (palavras-chave). Deve ser precedido da referência do documento.
\end{resumo}

% O resumo é obrigatório, em português e inglês. Este comando também gera
% automaticamente a referência para o próprio documento, conforme as normas
% sugeridas da USP
\begin{resumo}{eng}
While there is a wide variety of both open source and proprietary source code
static analyzers available in the market, each of them usually performs
better in a small set of problems, making it hard to choose one single
tool to rely on when examining a program. Combining the analysis of
different tools may reduce the number of false negatives, but yields a
corresponding increase in the number of false positives (which is
already high for many tools). An interesting solution, then, is to
filter these results to identify the issues least likely to be false
positives.

This work presents \textit{kiskadee}, a system to support the usage of static
analysis during software development by providing carefully ranked static
analysis reports.  First, it runs multiple static analyzers on the source
code. Then, using a classification model, the potential bugs detected by the
static analyzers are ranked based on their importance, with critical flaws
ranked first, and potential false positives ranked last.

To train kiskadee's classification model, we post-analyze the reports generated
by three tools on synthetic test cases provided by the US National Institute
of Standards and Technology. In order to make our technique as general as
possible, we limit our data to the reports themselves, excluding other
information such as change histories or code metrics. The features extracted
from these reports are used to train a set of decision trees using AdaBoost
to create a stronger classifier, achieving 0.8 classification accuracy (the
combined false positive rate from the used tools was 0.61).  Finally, we use
this classifier to rank static analyzer alarms based on the probability of a
given alarm being an actual bug. Our experimental results show that, on
average, when inspecting warnings ranked by \textit{kiskadee}, one hits $5.2$
times less false positives before each bug than when using a randomly sorted
warning list.
\end{resumo}

%%%%%%%%%%%%%%%%%%%%%%%%%%% LISTAS DE FIGURAS ETC. %%%%%%%%%%%%%%%%%%%%%%%%%%%%%

% Todas as listas são opcionais; Usando "\chapter*" elas não são incluídas
% no sumário. As listas geradas automaticamente também não são incluídas
% por conta das opções "notlot" e "notlof" que usamos mais acima.

% Listas criadas manualmente
\chapter*{List of Abbreviations}
\begin{tabular}{rl}
  NIST & National Institute of Standards and Technology \\
  SAMATE & Software Assurance Metrics And Tool Evaluation \\
  FP & False Positive \\
  TP & True Positive \\
  TN & True Negative \\
  FN & False Negative \\
  FLOSS & Free/Libre and Open Source Software \\
  CWE & Common Weakness Enumeration \\
  CVE & Common Vulnerabilities and Exposures
\end{tabular}

% Normalmente, "\chapter*" faz o novo capítulo iniciar em uma nova página.
% Como cada uma destas listas é muito curta, não faz muito sentido fazer
% isso aqui. "\let\clearpage\relax" é um "truque sujo" para temporariamente
% desabilitar a quebra de página.

%\chapter*{Lista de Símbolos}
%{\let\cleardoublepage\relax \addvspace{55pt plus 15pt minus 15pt} \chapter*{Lista de Símbolos} }
%\begin{tabular}{rl}
        %$\omega$    & Frequência angular\\
        %$\psi$      & Função de análise \emph{wavelet}\\
        %$\Psi$      & Transformada de Fourier de $\psi$\\
%\end{tabular}

% Listas criadas automaticamente
\listoffigures
%{\let\cleardoublepage\relax \addvspace{55pt plus 15pt minus 15pt} \listoffigures }

\listoftables
%{\let\cleardoublepage\relax \addvspace{55pt plus 15pt minus 15pt} \listoftables }
