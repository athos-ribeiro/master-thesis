%% ------------------------------------------------------------------------- %%
\chapter{Introduction}
\label{ch:introduction}

Any reasonably large program in production has bugs,
and minimizing their number and impact is an important aspect of software
development. However, finding bugs is often a difficult task, especially when
they do not manifest themselves directly but result in security flaws.

Automated static analysis tools, also referred to as static analyzers, can
alleviate this problem by examining the source code of computer programs for
pre-determined properties, particularly security flaws. In theory, source code
static analysis can explore abstractions of all possible program behaviors,
which is not feasible with software testing~\citep{hovemeyer_finding_2004}.
Hence, this class of tools has the potential to find problems that tend to
manifest themselves infrequently and are, therefore, harder to detect through
other methods~\citep{black2009static}, which makes their usage a complementary
approach to automated testing. Once aware of a possible anomaly, developers can
inspect the suggested section of the program and take action when necessary,
patching the flawed code even before it reaches production releases.

Such tools show great promise but, due to the undecidable nature of the
problem~\citep{landi_undecidability_1992}, static analyzers commonly use
heuristics or approximations to determine properties for a given code
construct, which frequently induces them to make inaccurate assumptions
regarding the behavior of the program under evaluation. These assumptions may
either lead the tool to generate a warning\footnote{When a static analyzer
finds an anomaly, it produces an \emph{alarm} or \emph{warning}. In this paper,
we use both terms interchangeably to refer to a location in the source code
pointed as a problematic site by a static analyzer. We use the term
\emph{report} to refer to a set of warnings generated in a single run of a
static analyzer for a given target program.} that does not reveal a real flaw
in the analyzed program, i.e., a \emph{false positive}, or to fail generating a
warning to a real flaw, i.e., a \emph{false negative}. Since static analysis
enumerates many execution paths, static analyzer reports frequently contain an
excessive amount of information, which often includes a substantial amount of
false positives due to the undecidability problem.

Empirical studies show that even carefully designed static analyzers may
generate warnings at false positive rates that can exceed
30\%~\citep{kremenek2004correlation}. Such false alarms require manual
inspection, increasing the effort of analyzing tool reports when they include
significant amounts of false
positives~\citep{muske2013review,heckman2007adaptively}, hindering the usage and
general adoption of automated source code analyzers~\citep{johnson_why_2013},
which can lead static analyzers to be discarded as
irrelevant~\citep{kremenek2003z}.
A substantial improvement in this area could foster
a more extensive inclusion of static analysis in software development cycles which,
in turn, could help reduce the number of bugs and security flaws in many different
kinds of software.

The mere reduction of the false positive rates obtained from static analysis
--- and, consequently, of the amount of counterproductive information generated
by static analyzers --- can smooth the process of manually verifying analyzer
reports. This idea has prompted the elaboration of several studies that propose
either pruning or ranking the warnings from static analysis reports to deal
with false
alarms~\citep{ruthruff_predicting_2008,kremenek2004correlation,heckman2009model,jung2005taming}.
These studies usually combine information extracted from both the static
analysis reports as well as the project being examined to derive conclusions
and infer if an alarm is an actual software flaw. In contrast, we hypothesize
that the incorporation of various specialist tools, i.e., the use of multiple
static analysis tools in conjunction, makes it possible to produce a reasonably
accurate predictive model without the need for any knowledge about the examined
source code.  This is advantageous because this model could serve as an entry
point for enhanced static analysis reports without the need of meticulous
pre-processing and training steps for each inspected program. Seeing that some
tools perform better than others in specific tasks due to different analysis
methods, the usage of multiple static analysis tools might also increase static
analysis coverage, reducing the number of false negatives that could occur by
using a single tool~\citep{black2009static}. Naturally, using multiple static
analyzers to inspect the same program source code also inflates the absolute
number of false positives. This downside, however, is an acceptable compromise
given the corresponding improvement in the \emph{rate} of
false positives --- the core of this work.

Furthermore, some static analysis techniques require significant processing to
perform the analysis~\citep{pan_bug_2006}, which makes the whole static analysis
process expensive. The latter issue also impacts research on static analysis,
since much processing is needed when one wants to analyze how different tools
perform on a large software corpus, study false alarm rates, or investigate
vulnerabilities behavior. \cite{johnson_why_2013} conducted a
survey to understand static analysis adoption on development processes.
The authors found that false positives and the way warnings are presented to the user are
some of the reasons why static analyzers are underused, and that improving the
integration of these tools to the development process with a more intuitive
presentation of flaws and better explanations could help increase their
adoption.  Automating the static analysis process would also enable the
creation of static analysis report databases that could be queried by
developers before new software releases and, if made public, would also provide
more information to researchers, enabling more studies, or allowing them to focus
on the actual research rather than waiting for static analyzers to complete
their analyses.


%% ------------------------------------------------------------------------- %%
\section{Objectives}
\label{sec:objectives}

In this study, we present \textit{kiskadee}, a system designed to support
\emph{continuous} static analysis in software repositories using multiple
static analyzers to generate, store, and rank warnings. Since each static
analyzer has its unique report format, kiskadee converts all reports generated to
a common report language. By running multiple static analyzers on the same code
base, kiskadee reduces the number of false negatives in the analysis. To
address false positives, kiskadee ranks warnings in the static analysis reports
using the AdaBoost algorithm's~\citep{freund1999short} classification
probabilities.  Warnings with the highest rank are more likely to indicate real
and more critical software flaws and warnings with the lowest rank are more
likely to be false positives. In this context, a warning is a single issue
produced by a static analyzer. Finally, kiskadee stores the ranked static
analysis reports in a database. The ranked reports in the database are made
available to kiskadee users, providing them with more accurate data. Moreover,
\textit{kiskadee} can track warnings by software version, providing developers
with a relatively small list of warnings to be investigated in a suggested
order, favoring the use of static analysis in in development processes.

%%%% <<<<<<

For our ranking experiments, we created and trained a
predictive model by processing a synthetic static analysis test
suite, which is a collection of source code snippets with specific flaws
injected in known locations. We ran multiple static analyzers using the test
suite as input and converted the analysis reports to a universal format to
ease further processing.
%since
%each tool usually reports warnings in its own unique format (it is easier work
%with the warnings once they are in a uniform, computer readable language).
Then, we checked whether every single warning matched one of the injected flaws in
the test suite, labeling each either as true positive or
false positive. After the labeling step, we extracted the
characteristics used to train our model. These did not include
any characteristics from the analyzed source code and project history, which
tend to be the most relevant ones when predicting warnings positiveness, as
shown in previous works~\citep{yoon2014reducing, jung2005taming}. To compensate
for this, we used an ensemble learning method~\citep{aima} to train several
weak classifiers, which were then combined to arbitrate about the positiveness of new examples.
Finally, instead of just classifying new examples as true and false positives,
we ranked the warnings based on their probabilities of being real flaws, where
the top entries were more likely to be of interest, and the last entries were more
likely to be false positives.

We guided this investigation based on the following research questions:

\begin{description}
  \item [RQ1:] \textit{Is it possible to rank or classify static analysis warnings without also
  manually inspecting and pre-processing the analyzed source code?} This would make
  such results more readily usable for a large number of projects.

  \item [RQ2:] \textit{Is it possible to partially or completely automate the process of
    labeling warnings as false positives or false negatives?} This would enable
    studies with large data sets in environments where manually inspecting
    warnings is not a frequent activity.
\end{description}

The results obtained were very promising: using three static analyzers with an
aggregate false positive rate of 0.61, we achieved an accuracy of 0.8, not too
distant from the state of the art (0.85), which depends on additional source
code analysis (such as extracting function names and sizes, and file names)
after the main static analysis step with the tools that generate the warnings
to be ranked.

\section{Contributions}
\label{sec:contributions}

This work provides the following three major contributions:

\begin{enumerate}[label=C\arabic*]
  \item \textbf{Continuous static analysis tool.}
    \textit{Kiskadee} is a tool able to continuously run different static\
    analyzers in different software versions by monitoring software\
    repositories through an extensible interface. It then stores the analysis\
    reports into a database, using a common reporting format. If a kiskadee's\
    production instance is made publicly available, its\ database may aid both
    open source software developers and researchers. Developers will be able to
    query\ warnings reported for their projects and act on possible flaws in
    their\ source code base. Researchers will be able to use the database as a\
    static analysis benchmark.

  \item \textbf{Method to rank warnings from different static analyzers based\
    on criticality and veracity of the reports.} By answering our research\
    questions, it will be possible to propose such method and rank\
    the warnings in the \textit{kiskadee} database, providing more value to\
    the output of static analyzers.

  \item \textbf{Static analysis warnings data set.} During this work, we also
    generated a publicly available data set of static analysis alarms from
    multiple static analyzers on a universal report format, generated by
    analyzing a synthetic static analysis test suite. This data set can be used
    to extend our work or aid different research on static analysis.

\end{enumerate}

The software packages produced during this research were released as open
source software. Links for resources, documentation, related papers and source
code are available at \url{https://pagure.io/kiskadee}. Kiskadee's development
was partially sponsored by the Google Summer of Code program under the Fedora
Project organization supervision. Students of the University of Brasília also
contributed to kiskadee source code through their \textit{Software Maintenance
and Evolution} course. We included appropriate acknowledgements and credits to each
kiskadee contributor in the \emph{Acknowledgements} section, at the beginning of
this thesis.

\section{Structure of This Work}
\label{sec:structure}

The rest of this thesis begins with a thorough description of the literature
review performed to situate our work with reference to the state of the art
regarding source code static analysis usage and warnings processing
(Chapter~\ref{ch:literature}). To get readers familiar with source code static analysis
terminology, Section~\ref{sub:terminology} defines or explains essential concepts. We
then have a closer view on the limitations in the quality of current static
analysis tools reports and on how previous works make good use of source code
analysis to minimize them (Section~\ref{sub:related_work}). As we will see, this entails
significant effort for every single program to be analyzed, highlighting the
utility of avoiding this step.  In Chapter~\ref{ch:kiskadee}, we introduce the continuous
static analysis tool developed during this work, provide details on its design,
and discuss the reasons behind our implementation decisions. This last
chapter presents our approach used to reduce false negative occurrences on
static analysis reports and to provide a database of static analysis reports for
different software versions, which is done by continuously running multiple
static analyzers on software repositories. Now, it remains to describe our
experience with the approach to mitigate false positive occurrences on static
analysis reports, which is left for Chapter~\ref{ch:ranking}. In Section~\ref{sec:data},
we talk about how we normalized the data obtained from the static analysis
tools for further processing and how we chose the features used to train our
ranking model. In Section~\ref{sec:positiveness}, we present our approach to
process this data and build this model. We finally proceed to describe and
discuss the experimental validation and obtained results in
Chapter~\ref{ch:results}.
